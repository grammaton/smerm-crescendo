%!TEX TS-program = xelatex
%!TEX encoding = UTF-8 Unicode
%!TEX root = 2020-GS-personal-statement.tex

%%%%%%%%%%%%%%%%%%%%%%%%%%%%%%%%%%%%%%%%%
% Journal Article
% LaTeX Template
% Version 1.4 (15/5/16)
%
% This template has been downloaded from:
% http://www.LaTeXTemplates.com
%
% Original author:
% Frits Wenneker (http://www.howtotex.com) with extensive modifications by
% Vel (vel@LaTeXTemplates.com)
%
% License:
% CC BY-NC-SA 3.0 (http://creativecommons.org/licenses/by-nc-sa/3.0/)
%
%%%%%%%%%%%%%%%%%%%%%%%%%%%%%%%%%%%%%%%%%

%----------------------------------------------------------------------------------------
%	PACKAGES AND OTHER DOCUMENT CONFIGURATIONS
%----------------------------------------------------------------------------------------

\documentclass[
	12pt,
	a4paper,
	%twoside,
	%twocolumn
	]{article}

\usepackage[T1]{fontenc}
\usepackage{Alegreya}

\linespread{1.05} % Line spacing - Palatino needs more space between lines
\usepackage{microtype} % Slightly tweak font spacing for aesthetics

\usepackage[english]{babel}

\usepackage[
	%hmarginratio=1:1,
	top=20mm,
	bottom=25mm,
	textwidth=17.2cm,
	columnsep=0.8cm%19pt
	]{geometry} % Document margins

\linespread{1.04}

\usepackage{paralist}

\usepackage{etoolbox}

\usepackage{
  fontspec,
	xltxtra,
	xunicode
}

\usepackage{
  graphicx,
  dblfloatfix
}

\AtBeginEnvironment{quote}{\small}

\usepackage[
	hang,
	small,
	labelfont=bf,
	up,
	textfont=it,
	up
	]{caption}

\usepackage{titling} % Customizing the title section
\usepackage{hyperref} % For hyperlinks in the PDF
\usepackage{booktabs} % Horizontal rules in tables

\usepackage{enumitem} % Customized lists
\setlist[itemize]{noitemsep} % Make itemize lists more compact

\usepackage{abstract} % Allows abstract customization
\renewcommand{\abstractnamefont}{\normalfont\bfseries} % Set the "Abstract" text to bold
\renewcommand{\abstracttextfont}{\normalfont\small\itshape} % Set the abstract itself to small italic text

\usepackage{titlesec} % Allows customization of titles
\renewcommand\thesection{\Roman{section}} % Roman numerals for the sections
\renewcommand\thesubsection{\roman{subsection}} % roman numerals for subsections
\titleformat{\section}[block]{\large\centering}{\thesection.}{1em}{} % Change the look of the section titles
\titleformat{\subsection}[block]{\large}{\thesubsection.}{1em}{} % Change the look of the section titles

\usepackage{fancyhdr} % Headers and footers
\pagestyle{fancy} % All pages have headers and footers
\fancyhead{} % Blank out the default header
\fancyfoot{} % Blank out the default footer
% Custom footer text

%----------------------------------------------------------------------------------------
%	TITLE SECTION
%----------------------------------------------------------------------------------------

\setlength{\droptitle}{-4\baselineskip} % Move the title up

\pretitle{\begin{center}\huge\bfseries} % Article title formatting
\posttitle{\end{center}} % Article title closing formatting
\title{Personal Statement} % Article title
\author{%
\textsc{Giuseppe Silvi}\\[1ex]% \thanks{A thank you or further information} \\[1ex] % Your name
%\normalsize Conservatorio S. Cecilia di Roma \\ % Your institution
%\normalsize \href{mailto:me@giuseppesilvi.com}{grammaton@me.com} % Your email address
%\and % Uncomment if 2 authors are required, duplicate these 4 lines if more
%\textsc{Jane Smith}\thanks{Corresponding author} \\[1ex] % Second author's name
%\normalsize University of Utah \\ % Second author's institution
%\normalsize \href{mailto:jane@smith.com}{jane@smith.com} % Second author's email address
}
\date{} % Leave empty to omit a date

%\renewcommand{\maketitlehookd}{%
%\begin{abstract}
%
%%\noindent Short summary of the proposed project (max.100 words) including main objective, research questions, methods, and relevance to the research at RITMO.
%\noindent\input{abstract.txt}
%
%\end{abstract}
%}

%----------------------------------------------------------------------------------------
%	DOCUMENT
%----------------------------------------------------------------------------------------

\begin{document}

\maketitle
%\raggedright
\thispagestyle{empty}
%----------------------------------------------------------------------------------------
%\section*{RESEARCH OUTLINE}
%Please upload a brief (1,000 words maximum) personal statement that:
%
%Explains your interest in this area
%Describes any relevant research experience - for example, as part of a previous degree
%Lists any academic work you have published or which is awaiting publication
I am writing to express my interests in the Doctoral Research Fellowship in Artificial Intelligence and Music (AIM), at Queen Mary University of London, as it has been my ambition to become a composer with a high scientific focus in my natural field of research: how we listen the time.

I had the Master of Arts in Electronic Music Composition at the Conservatory of Music Of Rome, Italy. During my master studies, I have developed personal research on listening issues, about the perceived differences of what is acoustical and what electroacoustic. This field of research pushed me to deeply understand the necessity of time variances in space, it led me to develop a loudspeaker to overcome some limits of traditional diffusion techniques.

My tactile approach at doing music was incentivized from my Teachers Giorgio Nottoli, Nicola Bernardini and Michelangelo Lupone. They are both composers and researchers and, as usual in the early days of electronic music, pioneers of technologic solutions to music composition. I was not supposed to build a loudspeaker. I built it to overcome certain limits of my music staging. It was my first research. It was pure magic.

In Italy, I had opportunities for Teaching Electronic Music at High School of Music and two different Conservatory of Music. I had very pleasure teaching to young students and educate them to better listening. I wish to improve my teaching career prospects. I really have a deep passion for research, continuous learning by doing-and-discuss in teamwork. I have learned to open my work to people, let them see what I am doing now, even the unsolved issues. I wish to engage a research team has a strong respect for open-learning and community sharing. I believe there is no better opportunity as the Doctoral Training in Artificial Intelligence and Music (AIM) available at The UKRI Centre to continue my studies and improve my skills. I believe UKRI Centre could enjoy my devoted-work capacity and passion.

Studying Artificial Intelligence and Music at Queen Mary University is an opportunity I would love to deeply dedicate myself. I hope during my stay in London I will able to contribute to mine and the others' knowledge in music perception. I am convinced that I will be a valuable addition to the programme. 
I hope to be given this chance.

Thank you for considering my application.\\
\bigskip

Yours sincerely \\
\indent Giuseppe Silvi

\end{document}
